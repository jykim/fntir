\documentclass[openany]{now} % creates the journal version
% \documentclass{now}  % creates the book pdf version

\usepackage{subcaption}

% a few definitions that are *not* needed in general:
\newcommand{\ie}{\emph{i.e.}}
\newcommand{\eg}{\emph{e.g.}}
\newcommand{\etc}{\emph{etc}}
\newcommand{\now}{\textsc{now}}
\newcommand{\newpar}{\bigskip\noindent}

\usepackage{color}
\newcommand{\authornote}[3]{\marginpar{\tiny\color{#1}#2: #3}{\color{#1}{$\star$}}}
\newcommand{\jin}[1]{\authornote{red}{Jin}{#1}}
\newcommand{\emine}[1]{\authornote{green}{Emine}{#1}}
\newcommand{\paul}[1]{\authornote{blue}{Paul}{#1}}
\newcommand{\note}[1]{\textit{(#1)}}

\title{Offline Evaluation for Information Retrieval}

% THe following author list is tentative
\author{
	Jin Young Kim \\
	Microsoft \\
	jink@microsoft.com
	\and
	Emine Yilmaz \\
	University College London \\
	emine.yilmaz@ucl.ac.uk
	\and
	Paul Thomas \\
	Microsoft \\
	pathom@microsoft.com
}

\begin{document}

% the following settings can be set or left blank at first
%\copyrightowner{N.~Parikh and S.~Boyd}
% \volume{1}
% \issue{3}
% \pubyear{2014}
% \isbn{978-0521833783}
% \doi{1234567890}
% \firstpage{23}
% \lastpage{94}

\frontmatter  % title page, contents, catalog information

\maketitle

\tableofcontents

\mainmatter

\begin{abstract}
Offline evaluation characterizes an information retrieval (IR) system
% based on human judgments
without relying on actual users in a real-world environment. \paul{This suggests that lab studies are in scope.}\jin{I think it's hard to draw boundaries, except for its goals.} Offline evaluation, notably test collection based evaluation, has been the dominant approach in IR evaluation and it is no exaggeration to say that shared evaluation efforts such as the TREC conferences have defined IR research over the years. The reason for this success lies in the ability to compare retrieval systems in a reusable manner.

%Recently, there has been several trends which necessitates the change in the role and method of offline evaluation.
Several recent trends however necessitate a change in the role and methods of offline evaluation. First and foremost, online search engines with large-scale user base has become commonplace, enabling online evaluation based on user behavior \paul{Doesn't this suggest offline evaluation doesn't matter? Tone this down?}\jin{We'll talk about its limitations later}. There are new endpoints for search, such as mobile phones and conversational agents, and the types of search results has diversified beyond a list of web documents to include other result types. Finally, crowdsourcing has provided ways for human judgments of any kind to be collected at a large scale. However, such online evaluation based on user behavior has its own challenges due to repeatibility as well the extensive amount of time needed to get online evaluation signals from the users. Furthermore, most smaller companies and academic researchers do not have access to such large scale user base. Hence, recent research in IR evaluation has focused on the advent of new  offline evaluation paradigms which are more user-centric, diverse and agile.

This survey aims to provide an overview of recent research in IR evaluation pertaining to the trends above. We first introduce offline evaluation for IR, focusing on how it relates to other evaluation paradigms such as online evaluation. We also overview traditional offline evaluation for IR, and how recent trends have shaped the research so far. We then review research in offline evaluation on three levels: human judgments, evaluation metrics and experiment design. This organization will allow readers to follow recent developments in research from micro-level (human judgment) to macro-level (experiment). Finally, we discuss evaluation practice in industry, which has been a major driving force in research and development in IR.

%\emine{It may be better to have an organizaiton like:
%	* Importance of offline evaluation
%	* Difference from online evaluation and why it is important
%	* Components of offline evaluation
%	* Finally, organization of the paper}
% \paul{We more or less have this in \S\ref{sec:evaluation-paradigms}, I think, although probably each of the sections in Ch1 could be expanded}
\end{abstract}

% \emine{Maarten mentioned that since this will be like a book we should not have statements like recently (if I remember correctly)}
% \paul{Hmm---tricky. Isn't this whole effort motivated by recent changes? How should we describe them?}


\chapter{Introduction}
\label{c-intro}

In this chapter, we survey the area and lay conceptual foundations for the rest of the paper. We first provide an overview of different approaches to IR evaluation. We then focus on offline evaluation, explaining traditional approaches and recent trends. Finally, we introduce a conceptual framework and the outline for the rest of this paper.

\section{Evaluation Paradigms in IR}
\label{sec:evaluation-paradigms}

Evaluating a search system, or any system that supports information access such as recommendation or filtering, is a complex problem. The performance of a search system is dependent on various contextual factors, such as the task at hand, the user's preference, abilities, location and other characteristics, and even the timing of the interaction. Also, the ultimate source of ground truth, the user's judgment, is subjective, volatile, and often hard to come by.

\subsection{Offline vs. Online Evaluation}

In order to meet these challenges, IR researchers have built a rich evaluation tradition. Most of this work has been based on a few simplifying assumptions. The document collection is static and the user's information need is represented as a description or a keyword query. The user's judgments in situ are replaced with judgments collected post-hoc and from third parties, often in the form of binary or numeric-scale labels.

We can define this evaluation paradigm as \textit{offline evaluation} \citep{INR-009} in that the evaluation of the system can happen without requiring an actual user. This makes offline evaluation particularly suitable for early-stage evaluation of an IR system, when users are hard to come by. Another typical characteristic of offline evaluation is that the test collection (a set of tasks, judgments and documents) is 'reusable', in that once built it can be used to evaluate new systems; because many factors are controlled, evaluations are also commensurable across time and between researchers.

An evaluation paradigm contrasting with offline evaluation is called \textit{online evaluation}. In a recent survey on this topic, online evaluation is defined as the evaluation of a fully functioning system based on measurement of real users' interactions with the system in a natural environment \citep{INR-XYZ}. That is, online evaluation directly employs user behavior in natural environment for evaluation.

As large-scale online services are commonplace now, online evaluation has become a viable option for companies with running services with large user bases. In the literature, there has been a plethora of papers on methodologies for online evaluation. While online evaluation has benefits in using data readily available as a by-product of serving users, this dependence on user behavior also creates limitations for online evaluation, which we will discuss later in this section.

\subsection{Hybrid Approaches}

So far we have introduced two evaluation paradigms -- offline and online evaluation -- with distinctive characteristics. Offline evaluation is based on human judges as substitution of real users, and has strengths in experimental control and reusability. Online evaluation is based on user behavior, and has strengths in fidelity and cost.

% \paul{really? How about ``\dots is based on abstractions of real users''?}

While these two approaches comprise the majority of evaluation efforts, there have been several approaches trying to find a middle ground. Click modeling \citep{chuklin2015click} and counterfactual online evaluation \citep{Li:2015, li2010contextual}, for example, re-use online user data for future evaluation. These approaches, while enabling the re-use of online user data, are still limited in that they are based on implicit signals from user behavior. For instance, it is not trivial to decide whether a user indeed found the clicked document relevant or not, even with all the contextual information.
\mdr{Not sure. You want to make this argument. The exact same thing can/should be said about expert judges. They are not not real users so their judgments aren't necessarily an indication that users will find a document relevant/useful.}
\jin{The argument has been softened a bit}
%\paul{how is that a limitation? need to be explicit}\jin{I tried to explain}

Another related line of work is user study-based evaluation \citep{Bron:2013, Liu:2014, Shah:2011}, which is widely used in interactive IR studies \citep{kelly2009methods}. In such work, a group of participants are typically brought into a lab environment and asked to perform a set of (usually predetermined) search tasks. It is common for this type of study to collect both behavior and labels from the participants to get a more complete picture of search activity. 

User studies bear similarities with offline evaluation in that they typically involve some form of explicit judgments, but their emphasis is more on understanding some aspect of users' search behavior, as opposed to comparative evaluation among search systems. Also, user studies tend to be limited in scale (typically less than 100 participants) and based on a biased, possibly not representative, sample of participants (typically people within the same institution).
\mdr{but then TREC is even worse off, just a handful of assessors...}
\jin{Crowdsourcing can help here...}
However, the distinctions are getting blurred as search engines increasingly serve more complex results, and SERP (search engine results page) or session-level evaluation is drawing more attention. In fact, some recent research has tried to use task completion settings for system-to-system comparison \citep{Xu:2009}. Also, crowdsourcing techniques are reducing barriers in getting access to a large number of subjects with diverse backgrounds. We will return to this point in Chapter~\ref{c-human-judgment}.

\subsection{When to Use Offline Evaluation}

\begin{figure}
	\centering
	{
	\includegraphics[width=1\textwidth]{images/online_vs_offline}
	}\vspace*{3em}
	\caption{Pros and cons of offline vs. online evaluation}
\label{onlinevsoffline}
\end{figure}
\mdr{please use normal latex tables}
%\paul{Figure~\ref{onlinevsoffline}---perhaps retype? The screen grab is pixellated. Also NB capitals in ``Cost/Reusability''}\jin{fixed the typo. Not sure what to do about pixellation, though}

At this point, a reader may ask: when should we use online vs. offline evaluation? While online metrics are certainly valuable and must-have when feasible, there are reasons we may need explicit input from human judges. 

Figure~\ref{onlinevsoffline} summarizes the advantages and disadvantages of online vs. offline evaluation. Offline evaluation typically requires access to explicit judgments of relevance obtained from relevance assessors, or judges. Obtaining judgments is an expensive procedure; hence, online evaluation tends to be cheaper compared to offline evaluation. Furthermore, online evaluation is based on signals that directly come from real users, which can enable us to get a more realistic signal of user satisfaction. 

On the other hand, online evaluation requires a running system as it is based on signals from real users. First, in initial stages of system development we simply might not have real users to study. Furthermore, small companies and academics may not have access to a large volume of users to be able to collect reliable signals. On the other hand, with the availability of various crowdsourcing services, it is relatively easy to collect labels from human judges.

Another major problem in online evaluation is that usually a significant amount of usage data is needed before one can reach reliable signals of satisfaction. \mdr{please qualify, see early work by Joachims} Hence, online evaluation tends to be very slow, which may not be suitable for evaluating the quality of new methodologies quickly. %On top of this, online evaluation necessitates the maintenance of a production system with a large user base along with experimental infrastructure, which is possible for large corporations but challenging otherwise.

More importantly, signals obtained from real users tend to be noisy and traces of user behavior are often insufficient to measure a user's true satisfaction. As an example, let's take clicks on results for evaluating a search engine. While a click is certainly an indication that the user is interested in the result, it is not clear whether the clicked result was actually satisfying. \cite{} Also, clicks are often concentrated on the top of the page regardless of result quality  \cite{} making them difficult to interpret. All in all, the ambiguity and bias inherent in user behavior often make it hard to infer the true quality of our products. \mdr{What is "true quality"? In the eyes of the users? Of the judges? Of the managers?}

Another consideration is the reusability of the data collected. In offline evaluation, typically the label is collected at the level of individual information item (i.e., document) and the system is evaluated by its ability to put more relevant items on top. This means the labels can be reused to evaluate new systems that produce different rankings of the same items. By contrast, the data collected from online system is typically valid for the evaluation of the system user interacted with, although there are new research to address these issues.

Offline evaluation, on the other hand, tends to be fast once explicit judgments of relevance are obtained from relevance assessors. Once these judgments are collected, they can be used to evaluate the quality of sytems quickly. This makes offline evaluation very suitable for trying new ideas, and the initial cost can be amortised over many experiments. %Furthermore, offline evaluation mechanisms can be used to evaluate the quality of new systems; hence, they tend to be reusable and portable.

One major drawback of offline evaluation is the expense of collecting these explicit judgments, or the ground truth. Obtaining relevance judgments can be slow and expensive, and has to be repeated if the notion of relevance changes. Furthermore, these explicit judgments of relevance tends to come from a third-party assessor, as opposed to the real user of the system. Hence, the assessor may have a different understanding than an actual user as to what documents should be considered relevant. Finally, depending on domain (i.e., medical or engineering) or task (i.e., personalized search), it is difficult to find capable assessors. \mdr{plus difficulties of getting expert labels in some domains/tasks (personalized search) plus dynamics of relevance}\jin{added}

Finally, offline evaluation metrics tend to be based on \emph{models} of user behavior, as opposed to behavior signals obtained from real users and modeling users can be quite challenging due to the variance in behavior and expectations of real users. Hence, evaluation metrics based on user models may not necessarily reflect user satisfaction. Much recent work in offline evaluation focuses on this issue, which we will review later in Chapter~\ref{c-human-judgment}. %\paul{where? need ref}

Given these advantages and disadvantages associated with online and offline evaluation, it is no wonder that most large scale companies tend to use a combination of both mechanisms, as we will go over in Chapter~\ref{c-practice}. Offline evaluation mechanisms tend to be used for quickly measuring the quality of new methods, and algorithms that show promising results are deployed to some selected subset of real users. Online evaluation mechanisms are then used to make the final decision about full-scale deployment. On the other hand, among smaller companies with limited users, and among academic researchers, offline evaluation remains the most viable mechanism.\jin{softened the argument}

\section{Offline Evaluation for IR}

Information retrieval has a rich tradition of evaluation, both online and offline, and this tradition has been responsible for some of the rapid advances in search technology of the past two decades~\citep{TRECimpact}. Below we survey traditional approaches to offline evaluation, and consider trends in recent years which suggest new roles and methods.

\subsection{Traditional Approaches to Offline Evaluation}

The traditional offline approach to IR evaluation is the test collection, or ``Cranfield'', approach first described by \cite{cleverdon67} and refined through exercises such as the Text REtrieval Conference (TREC; see \cite{voor:trec05} for an overview).  We will summarise this approach here, noting that \cite{INR-009} provides a historical summary and comprehensive discussion.

\begin{figure}
	%\centering
	{
	\includegraphics[width=0.67\textwidth]{images/ir-model}
	\subcaption{A simplified model of a retrieval system in context.}
	\label{fig:ir-simple}
	}\vspace*{3em}
	{
	\includegraphics[width=\textwidth]{images/ir-model-with-judges}
	\subcaption{Extension of the model, illustrating third-party relevance judges and the formation of a test collection.}
	\label{fig:ir-with-judges}
	}\vspace*{3em}
	\caption{Test-collection-based evaluation of an information retrieval system.}
\end{figure}
\mdr{That's a narrow view of IR: why not "provide the right information to the right person in the right way". So this could be about documents, but it could also be about answers, entities, etc.}\jin{done}
In its most basic form, we can think of an information retrieval system as providing the right information to the right person in the right way. A user has an information need; they express this as a query; and the system will draw on the collection of information to produce some set of results (Figure~\ref{fig:ir-simple}).

The test collection approach simulates this model by using judgements of relevance and evaluation metrics that aim at measuring the quality of the results presented to the user  (Figure~\ref{fig:ir-with-judges}). The query, document collection, and retrieval system are as before but three components are added:

\mdr{Should you not more explicitly and formally define what these core concepts are: need, query, inut for a need, test collection plus the relation between them?}

\begin{description}
	\item[Judges] interpret the user's information need, for example on the basis of the query or other context; and consider the extent to which each document in the collection answers this need.
	
	\item[Judgments] record this information obtained from the judges for each (query, document) pair.
	
	\item[Evaluation] is then a matter of aggregating the recorded judgements for the set (or ranking) of documents retrieved by the system; or comparing the documents retrieved with the documents judged as relevant.
\end{description}

For example, precision can be calculated by counting the number of retrieved documents which were marked as relevant; recall can be calculated by comparing the number of retrieved documents judged relevant with the overall number judged relevant; or rank-sensitive metrics, such as average precision or reciprocal rank, can look at the judgment for each retrieved document in turn.

\subsubsection{Abstractions}
\mdr{should you not put that in the big picture from the start? in a sense, we are sampling everywhere, queries, documents, judges. it would be good to point out the uncertainties that come with this from the start}
In principle, the judgements formed are complete---that is, the collection includes a judgment for each possible (query, document) pair. Although this was true of some of the earliest exercises \citep{cleverdon66}, it is clearly impossible for today's much larger document sets. Two shortcuts have allowed researchers to collect useful judgments regardless.  First, a simplifying assumption is that every search is over the same, fixed, document collection: that is, the documents do not change over time and nor are they different for different information needs (or users---as would be the case for personal or corporate collections).

Even allowing for this, it is clearly impossible to judge each of millions (or billions) of documents for any arbitrary information need. \emph{Pooling} provides a common shortcut. \cite{} In a pooled evaluation, each of a number of search systems provides its own ranking of documents, perhaps by running the same query on each. Every document which appears in the top~$k$ in any ranking is then judged, so if $N$~systems contribute to the pool there are at most $Nk$~judgments to be made: likely fewer, as some documents will appear in more than one list.

A related assumption is that each need can be captured in a single expression: that is, most (although not all) collections include a single query for each need.  Although this is clearly a very small sample of possible inputs for the need, and although relatively small changes to a query can result in large changes in measured effectiveness~\citep{bailey15user}, a large sample of needs can still capture useful variation.

Judging is also abstracted from real users in real contexts, in order to collect judgments at scale. The largest assumption here---indeed, an assumption relied upon by most effectiveness metrics---is that judgments are independent. That is, it is assumed that the extent to which a document is relevant to a need is independent of any other document which might be returned, or the order in which they are seen by the user. Notable exceptions are techniques from \cite{Golbus:2014:CDR} and \cite{Chandar2013}, who used relevance judgments based on other documents.

The notion of ``relevance'' is also normally abstracted. Although in reality relevance is complex, multi-facteted, and highly contextual \citep{borlund2003,saracevic16relevance}, judges are often given much simpler instructions which can for example boil down to what \cite{borlund2003} calls ``intellectual topicality'' alone. Recent work such as that by \cite{Mao:2016} and \cite{VermaYC16} also aims at extending this narrow definition of relevance. We delve into this in Chapter~\ref{c-human-judgment}.

\subsubsection{Test Collections}

A central concept in traditional IR evaluation is the \emph{test collection}. A test collection is the combination of 

\begin{itemize}
	\item a fixed set of documents; 
	\item a set of information needs or topics, typically each with an associated query; and 
	\item a set of relevance judgments which detail the relevance of at least some documents to each need.
\end{itemize}

Because they involve a static representation of an information-seeking session, test collections can be distributed;\mdr{what does that mean} the judgments therein are reusable; and, in combination with one or more effectiveness metrics, they make it simple to compare systems.
\mdr{Explain how your survey relates to Mark Sanderson’s survey and why we need your survey now.}
%\subsubsection{Building Collections}
%
%\paul{to do?} glossed over the many questions of how to build these in practice. simulated work tasks. query variation. pooling choices. labelling choices, judge selection (g/s/b). etc. see\dots

%\subsubsection{Shared Evaluation Exercises}

Test collections have been especially valuable for evaluation as they are easy to re-use: typically the limiting factor is just physical (or network) distribution of the documents themselves.\mdr{nope: pool bias} Since they are so abstracted, they are self-contained, and it is trivial to compare results across systems, times, or laboratories.

A noteworthy example of this is the Text REtrieval Conference (TREC) series, run annually by the (U.S.) National Institute of Standards and Technology (NIST). Since 1992 these conferences have been based around shared evaluations, using test collections so that each participating system can be directly compared to others~\cite{voor:trec05}. The model has been adopted by a number of other conferences including the NTCIR Workshop\footnote{\url{http://research.nii.ac.jp/ntcir/index-en.html}}, the Conference and Labs of the Evaluation Forum (CLEF)\footnote{\url{http://www.clef-initiative.eu/}}, and the Forum for Information Retrieval Evaluation (FIRE)\footnote{\url{http://fire.irsi.res.in/}}. These collections now include genera such as the web, microblogs, genomics, tourism, email, and others and in virtue of their scope and portability have become standard tools for information retrieval research.\mdr{Think of Hersh paper about relation between offline and user studies. Think also of relation between offline and online: often online is the key metric (REF?); e.g., work by Aleksandr Chuklin.}

\subsection{Recent Trends in Offline Evaluation}

So far we have looked at traditional approaches in IR evaluation. While this tradition has served the community well for the past few decades, there has been several trends which necessitate a change in the roles and methods of IR evaluation. In this section, we outline recent trends and discuss implications for offline evaluation.

\subsubsection{User-Centric Evaluation}
First and foremost, online search engines with large-scale user bases have become widely available and used, enabling online evaluation based on user behavior. This availability of user data has opened up the possibility of validating the assumptions of offline evaluation with actual user data. Recent work on evaluation metrics has embraced online user data to tune parameters of the metrics \citep[for example]{CarteretteKY11, Carterette:2012,smucker12stochastic,YilmazSCR10}.

The overall outcome of this trend is the advent of new IR evaluation paradigms which are more user-centric, diverse and agile. Here, being user-centric means that the evaluation process is based on a model of user behavior, or/and aims to improve user satisfaction or other user-visible measure such as engagement or task completion \citep{scholer13}. \mdr{this is a very diffuse definition of "user-centric, diverse and agile" evaluation. Can you split out the three notions and be more precise about the definitions of each? Also, what is a user-visible measure? Please define.}

This has already led to new methodologies to better estimate user satisfaction and behavior in judgment collection \citep{VermaY16, VermaYC16} or metric design \citep{YilmazSCR10, CarteretteKY11, ChapelleMZG09}. Also, some recent work has looked at cross-metric correlation, aiming to align IR evaluation with user satisfaction or some proxy of it \citep{Al-Maskari2007,radl:comp10}.\mdr{There is quite a bit more on this.}

%As a side note, there has been an increasing efforts to combine online and offline evaluation. These include ways to use online user data for offline evaluation \cite{Li:2015} \cite{li2010contextual} \cite{chuklin2015click}, or ways to collect feedback directly from user \cite{Kim2016}. 
%\note{mentions of user study / iir papers}

\subsubsection{Diverse Endpoints and Search Scenarios}

There are also new endpoints for search beyond desktop web browsers, such as mobile phones and conversational agents. This has opened up a whole area of research which focuses on different interaction methods and user experiences across endpoints. For instance, mobile devices have much smaller screen dimensions and the interaction is based on touch, while conversational agents use natural language, often in voice, to interact with the user.

Even for web search itself, the types of search results have diversified beyond the list of web documents to include other result types such as images, videos, news and even direct answers. This diverse set of results types, and corresponding user interface designs, breaks many assumptions of traditional IR evaluation, providing rich opportunities for exploration. In particular, many of these 'answers' can directly satisfy users' information needs on the SERP, making it hard to apply click-based evaluation techniques \citep{Li2009GA,diriye2012leaving}.\mdr{See Chuklin et al, CIKM 2016}
\mdr{I suggest organizing this differently: in every paragraph, first mention problem/challenge, then mention recent work that addresses this challenge, then scope: either point to later in the survey  in case you are addressing the problem or explicitly say that you are not addressing it}

IR evaluation research has with various lines of work. There has been increased interest in whole-page evaluation and optimization \citep{Zhou:2012}, which encompasses a wide variety of page elements beyond web results. %Side-by-side evaluation 
%
Task and session-level evaluation has also drawn interest \citep{KanoulasCCS11, CarteretteKHC14}, with TREC tracks of the same name \citep{carterette2014overview}. Finally, there have been new lines of work focusing specifically on mobile interfaces \citep{VermaYC16}, or evaluation of search with spoken agents \citep{Kiseleva:2016}.

\subsubsection{Crowdsourcing / Agile Evaluation}
\mdr{I don't see how the novelty or exploratory nature of new endpoints and scenarios calls for an agile manner of collecting labels. Vague.
	What do you mean "with less investments". Simply that it should be cheaper? 
	Or that TREC style judging does not scale for financial reasons?
	Aren't new devices and crowdsourcing orthogonal?}

These diverse new endpoints and scenarios for search required ways to collect labels in a more agile manner, because many of these services are new and exploratory by nature, with less investments compared to well-established ones like web search. Also, in academic settings, it has been difficult to recruit participants with diverse backgrounds at scale.

Fortunately, services such as Amazon Mechanical Turk have provided new ways for human judgments of any kind to be collected at a large scale. These services are called `crowdsourcing', in that they pull the `wisdom of the crowd' for tasks needing human intelligence. Accompanying this alternative data collection method is a challenge in quality control, since the labeling work is completed by a remote worker on the internet.

Given these opportunities and challenges, there has been quite a good deal of research on collecting high-quality labels with low effort \citep{Alonso2012}. Popular approaches include using overlapping judgments to identify ground truth labels \citep{Venanzi:2014}, or identifying the quality of judges based on their behaviors \citep{Kazai:2016}. We cover some of these methods in Section \ref{s-crowdsourcing}.

%\emine{Should we define that crowdsourcing is and how it may be useful for offline evaluation?}

\section{Scenarios for Offline Evaluation}

We have outlined basic concepts and recent trends for offline evaluation so far. The goal of this paper is to provide a practical guide to conducting offline evaluation for both academic and industry practitioners. Since there can be various scenarios in conducting offline evaluation, here we outline possible ones which we cover in this paper.

In traditional IR research, a typical evaluation scenario is to improve the performance of a document retrieval system given a test collection and a pre-determined set of evaluation metrics. For instance, in the TREC Web Track, participants are given a collection of web documents, and then asked to submit the results for their systems in a designated format. These are then evaluated on metrics like NDCG \citep{Jarvelin:2002} or ERR \citep{ChapelleMZG09}.

While academic IR research has developed well-accepted offline evaluation practices for document retrieval based on explicit labels, there are many evaluation scenarios not addressed in research from a practitioners' standpoint. There are multiple components in a modern IR system such as a web search engine, and designing evaluation for each requires different emphases and considerations. For instance, evaluating a query suggestion system can be quite different from evaluating a document ranking system. %, one can think of component-level (i.e., query suggestion) evaluation as opposed to system-level evaluation. 

Also, building a working system serving a large number of real users takes several stages of development. The evaluation at early stages of development would be more exploratory in nature, whereas at later stage the focus would shift to making ship decisions. We can call the former \textit{information-centric} evaluation in that the goal is to collect information helpful for system development and debugging, where the latter can be considered \textit{number-centric} in that the goal is to get reliable performance numbers for decision making.
%\paul{we don't use these terms again anywhere}

Another characteristics of IR evaluation in industry settings is that the evaluation is an on-going process which takes multiple iterations over the lifetime of the service, as opposed to a one-off research project. This necessitates the development of so called \textit{evaluation pipelines} where any new system can be evaluated on a ongoing basis.

Since the goal of this paper is to meet the need of practitioners as well as academic researchers, we describe decisions one needs to face in conducting offline evaluation across various scenarios outlined above. We also focus on considerations in designing a evaluation pipeline in industry settings at Chapter~\ref{c-practice}.

\section{General Framework for Offline Evaluation}

In this section, we describe in detail a framework for offline evaluation. The goal is to propose a general framework which can encompass the diverse set of scenarios outlined above. \mdr{Vague: "diverse set of scenarios outlined above"}

\subsection{Definitions}

First, here are a few definitions that will be used throughout this paper. These comprise the components of offline evaluation.

\paragraph{Search Task}  A search task is the user's information needs. And it is represented as a description or as a query.

\paragraph{Judging Target} Judging target denotes a result produced by an IR system, and the item which is evaluated. It can be of any granularity -- a snippet, a web document, or entire SERP. 

\paragraph{Human Judgment} A human judgment is an assessment of a \textit{judging target} by a human judge, in the context of a \textit{search task}, over some dimension of quality. 

\paragraph{Evaluation Metric} An evaluation metric (or metric in short) summarizes judgments into a single score. The design of an evaluation metric depends on the type of judgments being collected, and the model of user behavior.

\paragraph{Experiment} We define an experiment as a collection of search tasks, judging targets, and human judgments with a specific evaluation goal. An evaluation metric summarizes the outcome of an experiment with a test collection, and an appropriate statistical test can be used to make a claim about the validity and reliability of the findings.

\subsection{Evaluation Process}
Given the components above, here we discuss the general process for offline evaluation. At a high level, offline evaluation based on human judgment is composed of three steps: 1) judgment design, 2) metric design and 3) experiment design. Alternatively, you can consider the whole process in terms of collecting data (judgments), combining them into meaningful numbers (metrics), carry out experiments to test hypotheses and draw conclusions (test collection). Now we discuss major considerations in each step.%\paul{surely you need a collection, or at least 2/3 of it -- tasks and docs -- to get the judgements in the first place. Perhaps ``\dots, carry out experiments to test hypotheses and draw conclusions''?}

%\begin{figure}
%	\begin{center}
%		\includegraphics[scale=0.4]{images/offline_evaluation_overview}
%		\caption{Overview of Offline Evaluation.} 
%		\label{fig:offline_evaluation_overview}
%	\end{center}
%\end{figure}
%\emine{This figure shows as though the metric is part of the test collection so we might need to change that}
\subsubsection{Designing Human Judgments}

In the first step, the details of human judgment should be defined, which is the basic unit of offline evaluation. Human judgments capture the quality of the results for given search tasks. Here are major considerations in this step:

\begin{enumerate}
	\item How do you define and collect search tasks?
	\item What should be your judging unit?
	\item How do you design judging interface?
	\item How do you hire and manage judges?%\paul{we don't cover training in ch2}
\end{enumerate}

\subsubsection{Designing Evaluation Metrics}

The second step in offline evaluation is selecting or designing a  evaluation metric. Metrics summarize the information from individual labels into meaningful numbers. This is essentially the question of how to combine labels to meaningful numbers.

\begin{enumerate}
	\item How do you transform the labels from human judges?
	\item How do you define user models in combining labels into a metric? \mdr{You have not told us how or why user models enter the picture. Or what they are.}
	\item How do you estimate the parameters for the user model?
\end{enumerate}

\subsubsection{Designing Experiments}

%\paul{was ``designing test collections'', please check you're happy with the re-wording}
%\jin{Makes sense. ``Test collection'' seems a bit narrower.}

Lastly, judgments and metrics should be combined to achieve the goal of evaluation. Since this is an iterative step which takes several stages of refinement, here we describe methods and criteria in doing so. 

\begin{enumerate}
	\item How do you size the test collection to fulfill your evaluation goal?
	\item How do you evaluate the validity of the outcome?
\end{enumerate}


\section{The Organization of this Survey}

In the following chapters, we describe each process of offline evaluation in detail so that a reader can design his or her own evaluation pipeline following the flow of this paper. Chapter~\ref{c-human-judgment} deals with gathering judgments, which need to be created for the purpose. Chapter~\ref{c-metric} considers steps in designing an effective metric. Chapter~\ref{c-collection} covers the methods in designing and analyzing experiments. Finally, Chapter~\ref{c-practice} describes evaluation practices from major companies in search and recommendation area.
\mdr{How is this consistent with earlier statements and definitions that all seem to focus on *document retrieval*,}

%\emine{We already had a part describing the organization. In general, this section feels a bit repetitive given the text in first page}
%\paul{I disagree, that was in the abstract; it makes sense here (as well) I think. Unless I missed something?}


\chapter{Human Judgments}
\label{c-human-judgment}

The goal of collecting human judgments is to get an estimation of satisfaction of actual users of a search system by asking explicit questions to judges (or assessors), who simulate the actual users.  
%collecting a human judgment is to get an accurate measurement of  search engine results quality for given set of search tasks.
 A canonical example is collecting a binary relevance judgment for a document given a  search topic. The form of human judgments can be quite varied, however, depending on the type of search task and judging target.

We will start with an example to make the discussion more concrete. Figure~\ref{fig:human_judgment_overview} shows a list of possible search tasks about the topic of \textit{crowdsourcing} on the left side, and a few samples from existing web search results for query `crowdsourcing' on the right side.

\emine{Would it be better to show a more standard judging UI here? Like a query and a web page?}
\emine{I think this example is confusing as it is not clear what the task is; there are three tasks and it is not clear what the judge is judging. Why not use an interface where the task is clear and feedback mechanisms are also clear? This may confuse the reader since it doesnt show how the judgments are collected. }
\paul{Can we use another topic? We also discuss crowdsourcing below, which may be confusing. Let's use something which is clearly from a user not a researcher, e.g. ``rules of soccer''. I like the diagram otherwise I think}
\jin{@emine we do show judging I/F later. @paul let's keep it this way unless you strongly disagree -- i've used this topic throughout the chapter so it's not trivial to change}

\begin{figure}
	\begin{center}
		\includegraphics[scale=0.5]{images/human_judgment_overview}
		\caption{Overview of human judgment collection.} 
		\label{fig:human_judgment_overview}
	\end{center}
\end{figure}


This example presents basic ingredients in collecting human judgments -- search tasks and judging targets. From this example one can imagine a myriad of possibilities in designing a human judgment task. You can use either a (potentially ambiguous) keyword query or a well-defined topic description. You can collect judgment for a web document or any SERP element including instant answers or the list of news articles. 

The rest of this chapter is to give you guidance in designing a human judgment, in the light of recent literature on this topic. We will look over how to collect search tasks and how to determine a judging target. Various considerations in designing a judging interface will be examined, as well as the methods for finding and managing human judges.

%The first step in offline evaluation is collecting labels from human judges. In this chapter, we describe various considerations in collecting high-quality labels from human judges at scale. We first discuss the method for collecting search tasks, followed by the design of a judging method. We then discuss the collection of actual judgments, which is an non-trivial task to perform at scale. We also cover the trade-off and in using different types of judging resources -- in-house vs. crowd judges. (20-25 pages)


\section{Collecting Search Tasks}
Before considering judgment design, one needs to collect search tasks on which search results will be evaluated. Search tasks represents users' information needs that needs to be satisfied by the search results. In an industry setting where the search engine is used by actual users, the job of collecting search tasks can be as simple as sampling from queries users issued, whereas without access to such resources one needs to create tasks based on assumptions of target users and information needs. 

\subsection{Creating Search Tasks}
In many cases one needs to perform offline evaluation without a working system -- in building a new product, or in academic setting. It is essential to collect hypothetical search tasks in such cases, which is called simulated search or work (where work includes search and other things) tasks. \cite{Borlund:2003} summarizes the role of simulated work tasks as follows:

% consider non-simulated search tasks?

\begin{quote}
``A simulated work task situation, which is a short `cover story', serves two main functions: 1) it triggers and develops a simulated information need by allowing for user interpretations of the situation, leading to cognitively individual information need interpretations as in real life; and 2) it is the platform against which situational relevance is judged. Further, by being the same for all test persons experimental control is provided. Hence, the concept of a simulated work task situation ensures the experiment both realism and control.''
\end{quote}

`Task' can mean different things for different people, and IR literature has long debated over the definition of search task (see \cite{kelly2009methods} for a summary). For our purpose, it is sufficient to understand it as the representation of information needs which a human judge can use to perform a search and\paul{and/or? Often no searching is done by a judge} judge the quality of results.
\emine{Is that the definition of task? May be we should use a more proper definition?}
\paul{task $\neq$ need, but I think this use is blessed by so much past use}

The design of search tasks takes a few considerations which can critically affect evaluation results. First, there is the question of where the task is originated from and how much the judge is interested in or knowledgable about the task, or the corresponding domain. \cite{Edwards:2016} shows that judges' interests in the task has effects on how they perceive and perform the tasks. Judges in general had more knowledge on the tasks they were interested in, perceived the tasks as easier, and had higher engagement in terms of time spent. It is also known that (\cite{Bailey:2008}) judges' knowledge of the domain can affect the quality of the outcome.

Another dimension of task creation is the complexity, which again has many dimensions. \cite{Kelly:2015} looked at this problem using a cognitive complexity framework. They found that participants spent more effort (queries, clicks and time to completion) in performing tasks with higher cognitive complexity (create, evaluate and analyze) than tasks with lower cognitive complexity (apply, understand, remember).

In summary, these results show that the characteristics of search task is an important dimension in designing an offline evaluation. It is recommended to collect information about task characteristics and design experiments accordingly so that one can control the effect of these factors in evaluation.


\subsection{Sampling Query Logs}
Assuming you have a working search engine with real users, it is natural to collect search tasks from query log data. While this is a seemingly straightforward task, there are a few considerations. We outline some below, along with recommendations based on recent studies.

\paragraph{Evaluation Goals} The appropriate sampling strategy depends on evaluation goals. In a typical scenario, it is reasonable to start with a \textit{representative} sample of the traffic \paul{A random sample, you mean? Deduplicated? Balanced/stratified?}. Measurements based on this sampling strategy would lead to the characterization of \textit{average} performance, but there are scenarios where average performance is not informative. 

For example, a recent paper from  \cite{Zaragoza:2010} suggested techniques to identify segments useful for measurement. They introduce the notion of `disruptive sets', which are a set of queries with high quality results in one engine, but not in another. Using a disruptive set, one can focus on the set of queries with a goal to gain competitive advantage.

Other goals can also dictate the choice of sample. For instance, in industry one often targets a specific query segment (e.g., queries with fresh or local search intent); or perhaps on \textit{hard} queries where there is more room for improvement. In these cases sampling from the particular segment maximizes the evaluation efficiency.

\paragraph{Characteristics of Search Traffic} The characteristics of search traffic also needs to be considered. \cite{Baeza-Yates:2015} shows that web search query logs follow a power distribution, with longer tails. He suggests a sampling technique to generate a sample that follows this distribution. The main idea is to bin the queries based on the frequency, which allows the sampled queries to match the distribution of original query set. 

\paul{so, stratified and re-balanced?}
\jin{more details?}

\paragraph{Query vs. Task Description} While you can ask judges to imagine a search task given a query, it is open to question whether using query to represent an information need is optimal. Unlike search tasks, which should contain sufficient details of user information need, queries in a typical search engine are often in an abbreviated form, ambiguous and/or with typographical errors \cite{}. \paul{empty cite?}\jin{any recommended citation? i.e., \% of queries with errors}

These characteristics of user queries can be a significant source of noise because 1) there can be many query forms for the same information need~\cite{Bailey:2015:UVI}, and 2) inferring true information needs from queries can be hard. On the other hand, \cite{Yilmaz:2014:EID} argued that the choice of intent descriptions can also cause large variability in evaluation results and therefore the judging should be done based on queries.

All in all, despite limitations, user queries are still the most readily available sources of task information, and therefore are widely used for judging search results. One can mitigate the noise and ambiguity of the search query by training judges and presenting possible meanings of the query -- i.e., a SERP from a commercial search engine. We discuss this in detail in Section~\ref{s:judging-context}.

\section{Designing a Judging Interface}

Once the search tasks are collected, we are ready to design a system to gather judgments. There are several main considerations in designing a judging interface: we cover these in what follows.

\begin{enumerate}
	\item  How do we describe the context of a search task? \\(user location, preferences, previous queries in the session, etc.)
	\item  What should be the target of each judgment? \\(webpage, SERP elements or whole SERP)
	\item  What should be the scale of judgment? \\(absolute vs. relative)	
	\item  What is the quality dimension we want to measure? \\(relevance, usefulness, novelty, etc.)
\end{enumerate}

\subsection{Judging Context}
\label{s:judging-context}
There are many contextual variables that affect user satisfaction with any given search result: users' knowledge and preference, language, timing and location of the search, just to name a few. Even with well-defined search tasks, it is hard to specify all these factors, let alone with terse keyword queries. Providing some of this contextual information to judges can potentially reduce the user-judge gap, thereby increasing the judgment quality. \paul{refs?} 

The choice of what context to provide depends again on the evaluation goal -- what do you want judges to know about the search task? For instance, if you think user location is crucial in judging the relevance of results (which is the case in many tasks), you should present the user's location alongside the query text. Note that, if possible, the location information should be collected along with user queries to get a realistic sample of actual user locations.

Relevance judgments are also affected by what user already did during the session, so it is reasonable to present some part of user session as judging context. Several authors have examined this.\cite{Chandar2013} used a document as context, with the goal of collecting judgments when the context document has already been read. They proposed an evaluation framework for novelty and diversity evaluation. \cite{Golbus:2014:CDR} also experimented with using a document as a context, and found that the metrics based on conditional judgments correlate better with user preference at SERP-level. \paul{other refs?}\jin{on what?}

While one may assume that adding more and more context can only increase the quality of judgments, it should be noted that more context means more effort for judges in digesting and applying the information. Moreover, more context can increase judging cost by adding a further source of variability. That is, instead of collecting judgment for every search task, these judgments should now be collected for every query and context pairs, which can potentially make the evaluation prohibitively expensive. 
\paul{but you just suggested sampling e.g. location at the same time; so there'll be a 1:1 mapping query:context. But it's true that if you want to examine the effect of one more variable (market, location, time, \dots\ then you'll need more data)}\jin{Yes, judging based on query+context will add variance, which necessitates more data}

Therefore, one should carefully consider the cost/value trade-off in adding the context to a judging task. As an extreme example, \cite{Mao:2016} used the entire session as a judging context for collecting judgments on usefulness (as opposed to relevance) and found that usefulness metrics show higher inter-assessor agreement and better correlation with task-level user satisfaction. However, they recommend using usefulness evaluation only for post-hoc analysis of the experiments due to high cost associated with using the whole session as a context.

\subsection{Judging Target}
Judging target defines the basic form of judgment. In what granularity the judgment should be collected (judging unit), and whether the judgment should be given for a single item, or a set of items (judgment type) should also be decided while creating a judging interface. 

\subsubsection{Judging Unit}
Judging unit defines the unit at which judgment should be collected: i.e., in what granularity do we want to collect judgments? In web search, for example, judging unit can be a webpage, SERP elements or a whole SERP, as shown in Figure \ref{fig:judging_units}. 

\begin{figure}
	\begin{center}
		\includegraphics[scale=0.5]{images/judging_units}
		\caption{Various judging units for web search results.} 
		\label{fig:judging_units}
	\end{center}
\end{figure}

Basically, judging unit should be determined by the goal of evaluation: if you care about the quality of ranked list, collecting judgment for each web search result seems like a natural choice. If the presentation of the whole SERP is primary concern, the entire SERP should be the right unit for judgment. 

On the other hand, if the judging target is reasonably complex with multiple sub-components, it is also possible to collect judgments at smaller units (i.e., SERP elements) and then calculate scores for large unit (i.e., whole SERP) by combining unit scores in a sensible way. This is how most of IR evaluation metrics (i.e., MAP or NDCG) works.

Now, if we want to collect judgments for SERPs, should we collect element-wise judgments and then combine, or collect single SERP-level judgments? This question can be generalized into the decision of judging unit when the judging target is complex. In fact, there is no hard and fast rule in determining right judging unit, but here we describe a few trade-offs. 

Smaller judging unit means simpler judging task which can be faster and more reliable individual judging task. However, the number of judgments to evaluate larger judging unit (i.e., SERP) can be quite high if the judging unit is small, making overall judging cost higher than collecting a single judgment for larger judging unit.

Smaller judging unit also means better reusability of individual labels, because you can combine labels for each SERP element to evaluate arbitrary SERP configuration. This means that the cost of collecting judgments can be amortized over multiple experiments. In fact, query-URL relevance judgment has been so widely used in TREC and other settings because it allows the creation of test collection which can be used to evaluate any ranked list.

On the other hand, smaller judging unit makes an assumption that each label can be collected independent of other element. This is hardly true in a typical search scenario where the concept and criteria of relevance can evolve over time. On this regards, larger judging unit has the benefit of providing rich context for judges. Also, larger judging unit can capture the interaction between elements -- i.e., redundancy among documents in a ranked list.

In literature, as briefly mentioned above, document-level judgment is most prevalent. However, there has been a few papers which deal with SERP-level evaluation. \cite{Bailey2010} introduces a judgment scheme which can capture the interaction among SERP elements as well as element-level quality. 

SERP-level judgments were introduced in \cite{Thomas2006}, where they used pairwise judging in order to minimize the complexity of defining judging criteria. (more about this in the following section) Several other works including  \cite{Kim:2013} refined this idea to include dimensional relevance judgments as well as overall SERP-level comparison.
% \cite{Al-Maskari2007} and
\paul{Add work by Falk et al.\ on judging snippets}

\subsubsection{Absolute vs. Relative Judgment}
Another consideration in determining a judging target is the type of judgment, which can be either absolute or relative. In absolute judgment judgment is collected for a single judging target, whereas relative judgment asks for pairwise preference between two judging targets. Figure \ref{fig:judgment_types} shows two types of judgments in evaluating web search results.

\begin{figure}
	\begin{center}
		\includegraphics[scale=0.5]{images/judgment_types}
		\caption{Absolute vs. Relative judgments} 
		\label{fig:judgment_types}
	\end{center}
\end{figure}

\emine{In Figure 2.3 here  we show a ranked list of results and ask the user how they rate the search result, which is confusing. I think this should either be individual document or ask a different question for the whole page}

Now, how should one choose between absolute vs. relative judgment? In general, making an absolute scale judgment requires having objective criteria among different levels, whereas relative judgment can avoid the issue. \cite{CarteretteBCD08} also suggested that relative judgments tend to be more accurate for document-level judging. \cite{Kazai:2013} also found that a pairwise judging mode improves crowdsourcing quality close to that of trained judges.

Relative judgments have been used in various evaluation settings. \cite{Chandar2013} employed document-level pairwise judging using another document as a context, with a goal of novelty and diversity evaluation. \cite{Arguello:2011} also proposed an evaluation scheme for aggregated search based on pairwise preference judgment at element-level. \cite{Zhou:2012} used SERP-level pairwise preference judgment as a part of the evaluation framework for aggregated search.

On the other hand, absolute judgments are reusable in that you can compare among any items for which you have item-level labels; whereas in the case of relative judgments you need to collect labels for every pair of items. Therefore, if you want to reuse judgments in a production environment where multiple generations of ranking techniques should be compared against each other, absolute judgments might save the cost in the long run. This is also the reason that TREC has employed absolute judgment since its inception.
\paul{Add work by Falk et al.\ on magnitude estimation}
\paul{Add work by Diane et al.\ on the effect of question mode? Can't remember if this is relevant}


\subsection{Judging Criteria}
The central assumption of offline evaluation is that human judges can represent real users, and we often want judges to tell us if the judging target would be relevant to the potential user. However, this is not a trivial task for judges given contextual and multi-faceted nature of relevance. (\cite{Borlund:2003}) Actually, \cite{Chouldechova:2013} reports increased judging quality when done by query owners (users who did the search themselves) compared to query non-owners.

Also, while the concept of relevance is broad, it typically specifies the relationship between an information need and an object, and is not sufficient to capture the true value of the item in the context of search session. Therefore, it has been argued \cite{Belkin:2015:SAL} that IR as a field should move beyond relevance to evaluate usefulness in the context of search task. TREC Session track \cite{carterette2014overview} is another movement in the same spirit.

\emine{Is there a reason why we used session track here but not tasks track? Tasks track used usefulness based judgments and focuses on tasks}

\jin{Relavance seems to subsume usefulness according to Borlund:2003. But Belkin:2015 seems to use a narrow definition of relevance.}

Recent work have tried to address this problem from multiple angles. The role of user effort and effort-based judging has been proposed \cite{Yilmaz:2014} \cite{VermaYC16}, where it is shown that effort should be incorporated as an additional factor in human judgment to build retrieval systems that optimize user satisfaction. \cite{Golbus:2014:CDR} and \cite{Kim:2013} also experimented with multi-dimensional judgment collection, which is useful in finding the relationship between different aspects of relevance.

Another thread of work looked at the relevance judgment in the context of other document, or even the whole session. \cite{Chandar2013} proposed judging methods for novelty and diversity, where they employed preference-based judgment between document A and B in the context of another document (C). The resulting method has benefit of allowing the evaluation of novelty and diversity without requiring the collection of sub-topical judgments. 

\cite{Mao:2016} proposed to collect usefulness judgment in the context of whole session. They showed that high relevance by assessors is a necessary but not sufficient condition for high usefulness for users, and that usefulness judgments better correlates with behavioral signals such as click cumulative gains. But since usefulness judgments are costly to collect, they advised the usefulness judgments for use in post-hoc evaluation.

In overall, current literature suggests many ways to set judging criteria for relevance, with different methods having different emphases. If the goal is to focus on query-document relevance, a simple interface as seen at the top of Figure \ref{fig:judgment_types} will do. However, one can add another document or even whole session history as a context if the goal is  to capture the value of the item in the context of a search task. 

\section{Collecting Judgments}

Once you have judging interface, now you need to find judges to work with. There are quite a few options from which you can find judges, but you can roughly put them into three categories: 1) team members who work on the project, 2) expert judges who typically sit in-house with the team, 3) crowd judges who work remotely and can be reached via platforms like Amazon Mechanical Turk. 

How should we decide on which option to choose? First, it is recommended to start some judging exercise with the team (Group 1) before outsourcing the judging task, because you need to make sure you provide high-quality interface and descriptions to get judgments of reasonable quality. But this approach soon hits scalability issues, so we focus on expert judges (Group 2) and crowd judges (Group 3) in this paper.
\emine{Why do we have to start with the team? Why does it have scalability issues? This part is not clear to me}

There has been some recent work comparing human judges of different characteristics. \cite{Bailey:2008} is a classic work where they found that judges' level of expertise on the domain can result in small yet consistent difference on system scores and rankings. Similarly, \cite{Chouldechova:2013} looked at judgments done by query owners (users who did the search themselves) vs. query non-owners, where they concluded that query owners are can distinguish a higher quality set of search results from a lower quality set in a blind comparison.

However, neither finding domain experts nor using queries done by judges themselves are feasible if you need judgments at scale, or need to collect judgments from representative sample from traffic. Typically the options available are either in-house judges with some training or crowd judges. Among these groups,  \cite{Kazai:2013} found that trained judges are significantly more likely to agree with each other and with users than crowd workers. But when they compared to judgments with clicks from real users, they found that the judgments from trained judges does not show higher agreement with user clicks.

\subsection{Crowdsourcing Relevance Judgments}

Crowdsourcing has an unparalled benefit in cost and scalability, and it has gathered a lot of attention from research community, and a large body of work has been produced in IR community as well. \cite{Alonso2012} provides a comprehensive survey of research and best practice in this area. 

Since aggregating redundant judgments from a group of independent assessors has been standard approach in reducing errors, some of these work have focused on collecting and aggregating redundant labels. \cite{Venanzi:2014} proposed a community-based Bayesian label aggregation model which is based on finding latent groups among crowd workers and aggregating labels based on them. \cite{Davtyan2015} proposed using textual similar to aggregate crowd judgments, where the relevance labels from similar documents are propagated. Companies such as Crowdflower \footnote{https://www.crowdflower.com/} provide services by which high quality labels are automatically calculated based on redundant judgments.

Another approach in improving the quality of crowdsourced judgments is by improving the judging interface design workflow by which crowd judges work on judging work. This section already dealt with design decisions on judging interface design, and \cite{Kazai2012} provide further guidance in deciding the complexity of judging tasks and the amount of payment per judgment.

Several recent work has investigated workflow design for crowdsourcing. At microscopic level, \cite{Shokouhi:2015} and \cite{Scholer:2013} looked at the effect of previous assessment on the quality of current judgment, where they showed that the human annotators are likely to assign different relevance labels to a document, depending on the quality of the last document they had judged for the same query. At macroscopic level, \cite{Megorskaya2015} explored various parameters in designing workflow, where they argued for having a communication channel between judges and the overlap of 3 -5 for production environment.

\section{Open Issues}

So far in this section, we looked at issues in collecting human judgments, and provided guidances based on latest research. However, the problem of search is rapidly evolving and as such emerging are new areas for research. Before moving on to the next topic, here we discuss several open issues.

\paragraph{New Judging Target} Most of existing research considers document-level judging. But modern SERP contains rich results beyond documents such as instant answers and multimedia results. Extending document-based judging model into these new judging targets would be an interesting problem. This include judging method for captions, instant answers and rich SERPs with all these elements.

\paragraph{New Endpoints for Search} Smart phones are becoming a standard devices for accessing the internet; and recently conversational agents have become a major focus for many tech companies. We are still yet to find how these new environments of information access can affect judgment collection. Recent work such as \cite{VermaY16} and \cite{Kiseleva:2016} provide precursor in what needs to change for these new environments.

\paragraph{New Judging Methods} Standard judging methods collect labels given a search task and a pair of searc results. However, this model may not work in environments where search is highly contextual and personal. Several recent work such as  \cite{Moraveji:2011} and \cite{Xu:2009} explored task-based judgment collection, where judges perform search given a search engine to make their judgments.


\chapter{Evaluation Metrics}
\label{c-metric}

The second step in offline evaluation is selecting or designing a meaningful evaluation metric. This is essentially the question of how to combine labels to meaningful numbers. For traditional IR evaluation where the labels are collected at query-URL level, combining labels to a metric requires quite a few assumptions, or even a user model. In this chapter, we go over the various considerations of IR metric design, as well as the user models behind these metrics. We briefly survey some established metrics but spend more time on recent developments: explicit models of user behavior, deriving metrics from these, and open issues including session-level measurement, dealing with variation, and considering rich SERPs. (20-25 pages)

\section{Basic IR evaluation metrics}

- Metrics based on absolute judgments (e.g. \cite{cooper73selecting})

- Metrics based on preference-based judgments, including e.g.\ aggregated in-situ side-by-side \cite{Thomas2006}

- Ranking-based metrics (Tau/TauAP)

- Criticisms: especially reproducability/replicability

\section{Metrics based on simple aggregation of labels/qrels}

- Set-based: P, R

- Rank-based: P@$k$, AP, RR

- Criticisms: what tasks and behaviors are modeled here?

\section{Models of behavior}

Evaluation metrics that are based on explicit models of user behavior

- The cascade model and variants

- Weights, the C/L/W framework \citep{Moffat2013}

- ERR, EBU, GAP, Time-biased gain, etc.

- Weighted precision metrics such as RBP, INST; notion of residual \citep{Moffat08,Moffat15}

- $\alpha$-NDCG, IA metrics, etc.

- Cost-based/economic models and the prospects of metrics from these

- Session-level metrics \cite{kanoulas2011evaluating} \cite{Järvelin2008}

\section{Model fitting}

Fit of metrics to models; estimating the distribution of parameters/metric values based on user data

\cite{CarteretteKY11}, \cite{Moffat2013}

\section{Open issues}

Open issues in behavior models and the corresponding metrics

- Whole-page quality

- Caption effects

- Variation between users: behaviors, learning styles, cognitive styles, topic expertise, search system expertise, expectations of the system, query variation, \dots

- Duplication in SERPs

- Learning (?)

- Non-traditional tasks and novel UIs

- Choosing between metrics; sensitivity; finding evidence any of them correlates with user behavior or other important dependent variables

- Measuring things outside the SERP: query formulation, source/engine selection

\chapter{Test Collections}
\label{c-collection}

Experiments is defined as the collection of labels and metrics defined on top of them. We first look over many considerations in order to design an experiment given a budget and time constraint. We then focus on a set of analyses we can perform once the data is collected, followed by the ways of reporting experimental results. (\ensuremath{\approx} 15 pages)

\section{Designing an Experiment}

- How to select queries?

- How many queries? \cite{Sakai:2014}

- How many documents? \cite{CarterettePFK09}

- How to distribute judgment efforts across queries and documents? \cite{CarterettePKAA09, YilmazR09}


\section{Analysis of Experimental Results}

Survey of research results \cite{Sakai:2016}

Drawing conclusions from metrics 

- Hypothesis Testing \cite{Dincer:2014}

- Comparison of different types of significance tests \cite{SmuckerAC09}

\newpar
Various analysis methods

- Power analysis \cite{Sakai:2014}

- Failure analysis

- Sensitivity and Reliability analysis \cite{Urbano:2013} 

- Informativeness (MaxEnt) \cite{AslamYP05}

- ETC \cite{Bron:2013} \cite{Boytsov:2013}  \cite{Robertson:2012}

\newpar
Reporting results

- Effect sizes and distributions, vs point estimates and $p$ values

\section{Open Issues}

- Reusability for SERP/task-level evaluation

- Beyond significance testing -- bayesian alternatives?

- Reusability / Generalizability of experimental results


\chapter{IR Evaluation in Practice}
\label{c-practice}

In this chapter, we review evaluation practices happening in both academia and industry. We first cover evaluation practices from academia, including recent TREC tracks, data generation efforts. We also look at evaluation efforts in related area such as recommendation systems and conversational agents. We then turn to evaluation practices from industry including major players in search and recommendation based on published papers and articles.

\section{Evaluation Practices from Academia}

Emerging TREC tracks

- Task track

- Microblog track

- Live QA track

- Contextual suggestions track

\newpar
Dataset generation efforts

- Living labs for IR \footnote{http://living-labs.net/}

- Data set shared by industry \footnote{http://jeffhuang.com/search\_query\_logs.html}

\newpar
Evaluation in related domains

- Aggregate search \cite{Zhou:2013}

- Recommendation systems \cite{gunawardana2015evaluating}

- Conversational agents

\section{Evaluation Practices from Industry}

How are the practitioners doing? (\ensuremath{\approx}15 pages)

- Google \footnote{How Search Works (Google) https://www.google.com/insidesearch/howsearchworks/thestory/} \footnote{Updating Our Search Quality Rating Guidelines
	 https://webmasters.googleblog.com/2015/11/updating-our-search-quality-rating.html}

- Bing \footnote{The Role of Content Quality in Bing Ranking (Bing)
	 http://bit.ly/1T1BaYN}

- Netflix \cite{Gomez-Uribe2015}  \footnote{The Netflix Tech Blog: Learning a Personalized Homepage
	http://techblog.netflix.com/2015/04/learning-personalized-homepage.html}

- Facebook \footnote{Who Controls Your Facebook Feed (Slate) http://slate.me/1T1BbvU}

- Pinterest \footnote{Machine Learning at Pinterest http://www.slideshare.net/HiveData/the-hive-think-tank-machine-learning-at-pinterest-by-jure-leskovec-61383413}

- LinkedIn \footnote{http://www.slideshare.net/dtunkelang/search-quality-at-linkedin}

- Startups \footnote{The Humans Hiding Behind the Chatbots http://www.bloomberg.com/news/articles/2016-04-18/the-humans-hiding-behind-the-chatbots}

\footnote{10 Data Acquisition Strategies for Startups http://bit.ly/28IHlC7}

\newpar
Common features: combine online and offline evaluation

- Offline evaluation for early iteration

- Online evaluation for final ship decisions

\chapter{Conclusions}

In this chapter we conclude this survey by providing the summary of contents so far. 
We also provide a brief outlook toward the future of offline evaluation for IR.

\section{Summary}

Recap: general Components of Offline Evaluation

-	Experiment

-	Search Task (Query / context)

-	Evaluation Metric

-	Judging Method (Interface / rating scale) 


\section{Future of Offline Evaluation for IR}

Emerging trends in the tech ecosystem

- Mobile-first: different interfaces and information needs

- Natural-language interaction: Bots and Conversational agents

- End-to-end support for task completion: e.g., restaurant reservation 

\newpar
Future of Offline Evaluation

- Evaluation of search agents (as well as engines)

- Evaluation of various information 'cards'

- Evaluation of end-to-end task completion

\newpar
Future of Offline Evaluation Research

- Need for Academy-Industry collaboration

\backmatter  % references

\bibliographystyle{plainnat}
\bibliography{ch1_intro,ch1_user_study,ch2_judgment,ch2_crowdsourcing,ch3_metrics,ch4_experiment,ch5_industry}
	
\end{document}
