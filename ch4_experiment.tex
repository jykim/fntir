\chapter{Experiments}
\label{c-collection}

An experiment is defined as the collection of labels and metrics defined on top of them. We first look over many considerations in order to design an experiment given a budget and time constraint. We then focus on a set of analyses we can perform once the data is collected, followed by the ways of reporting experimental results. (\ensuremath{\approx} 15 pages)

\section{Designing an Experiment}

- How to select queries?

- How many queries? \cite{Sakai:2014}

- How many documents? \cite{CarterettePFK09}

- How to distribute judgment efforts across queries and documents? \cite{CarterettePKAA09, YilmazR09}


\section{Analysis of Experimental Results}

Survey of research results \cite{Sakai:2016}

Drawing conclusions from metrics 

- Hypothesis Testing \cite{Dincer:2014}

- Comparison of different types of significance tests \cite{SmuckerAC09}

\newpar
Various analysis methods

- Power analysis \cite{Sakai:2014}

- Failure analysis

- Sensitivity and Reliability analysis \cite{Urbano:2013} 

- Informativeness (MaxEnt) \cite{AslamYP05}

- ETC \cite{Bron:2013} \cite{Boytsov:2013}  \cite{Robertson:2012}

\newpar
Reporting results

- Effect sizes and distributions, vs point estimates and $p$ values

\section{Open Issues}

- Reusability for SERP/task-level evaluation

- Beyond significance testing -- bayesian alternatives?

- Reusability / Generalizability of experimental results


